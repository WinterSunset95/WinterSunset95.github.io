\documentclass{article}
\usepackage{amsmath}
\title{Day 2}
\author{WinterSunset95}
\date{10th May 2025}
\begin{document}
\maketitle
\section{Index}
\begin{enumerate}
	\item A1
	\begin{enumerate}
		\item Determinants
		\item Minors and Cofactors
		\item Application of Determinants
		\item Inverse of a Matrix
		\item Solving simultaneous linear equations
	\end{enumerate}
	\item B2
	\begin {enumerate}
		\item Allegation and Mixture
		\item Basic numerical problems
	\end{enumerate}
\end{enumerate}

\section{Calculations}
\[
	P = \begin{bmatrix}
		-3 & 4 \\
		2 & -5
	\end{bmatrix}
\]
	\begin{align*}
		|P| &= ad - bc \\
			&= -3(-5) - 4(2) = 7 \\
	\end{align*}

Ex. Area of triange with vertices P(0,1), Q(2,3), R(4,0) \\
Solution: The matrix A formed by the coordinates is 
\[
A = \begin{bmatrix}
		0 & 1 & 1 \\
		2 & 3 & 1 \\
		4 & 0 & 1
	\end{bmatrix} \\
\]
Now, \\
\begin{math}\begin{aligned}
|A| &= a_{11} c_{11} + a_{12} c_{12} + a_{13} c_{13} \\
c_{11} &= (-1)^{1+1} det\begin{bmatrix}
						3 & 1 \\
						0 & 1
					\end{bmatrix} \\
	&= 1x(3x1)-0 = 3\\
c_{12} &= (-1)^{1+2} det\begin{bmatrix}
						2 & 1 \\
						4 & 1
					\end{bmatrix} \\
	&= -1(2-4) = 2\\
c_{13} &= (-1)^{1+3} det\begin{bmatrix}
						2 & 3 \\
						4 & 0
					\end{bmatrix} \\
	&= 1(0-12) = -12\\
So, \\
|A| &= 0x3 + 1x2 + 1(-12) \\
	&= 2-12 = -10 \\
\end{aligned}\end{math}
Therefore, \\
Area = 1/2|A| = 1/2(-10) = -5 \\

Ex. Find the inverse of A = \begin{bmatrix}
							3 & 1 \\
							4 & 2
						\end{bmatrix} \\
Solution: Firstly, \\
\begin{math}\begin{aligned}
	|A| &= ad - bc \\
		&= (3x2) - (1x4) = 6-4 = 2 \\
\end{aligned}\end{math} \\

Here, |A| != 0, therefore A' exists. \\
\begin{math}\begin{aligned}
adj(A) &= \begin{bmatrix}
			d & -b \\
			-c & a
		\end{bmatrix} \\
	   &= \begin{bmatrix}
			2 & -1 \\
			-4 & 3
		\end{bmatrix} \\
\end{aligned}\end{math}
Now, \\
\begin{math}\begin{aligned}
	A' &= \frac{1}{|A|} adj(A) \\
		&= \frac{1}{2} \begin{bmatrix}
			2 & -1 \\
			-4 & 3
		\end{bmatrix} \\
		&= \begin{bmatrix}
			1 & -\frac{1}{2} \\
			-2 & \frac{3}{2}
		\end{bmatrix} \\
\end{aligned}\end{math}

\begin{bmatrix}
	3 & 1 \\
	4 & 2
\end{bmatrix}
\begin{bmatrix}
	x \\
	y
	\end{bmatrix} = \begin{bmatrix}
	7 \\
	10
	\end{bmatrix}

\end{document}

I did it this way, letting p be Priya's current age and r be Rina's current age.
so p/r = 3/4
=> p = 3r/4

now, 5 years ago, their ages were in the ratio 5/7
=> (p-5)/(r-5) = 5/7
=> ((3r)/4 - 5)/(r-5) = 5/7
=> (3r-20)/4 = 5r-25)/7
=> 7(3r-20) = 4(5r-25)
=> 21r - 140 = 20r - 100
=> r = 40

now, 
p/r = 3/4
=> p = 3r/4 = 120/4 = 30
